% Options for packages loaded elsewhere
\PassOptionsToPackage{unicode}{hyperref}
\PassOptionsToPackage{hyphens}{url}
%
\documentclass[
]{article}
\usepackage{lmodern}
\usepackage{amssymb,amsmath}
\usepackage{ifxetex,ifluatex}
\ifnum 0\ifxetex 1\fi\ifluatex 1\fi=0 % if pdftex
  \usepackage[T1]{fontenc}
  \usepackage[utf8]{inputenc}
  \usepackage{textcomp} % provide euro and other symbols
\else % if luatex or xetex
  \usepackage{unicode-math}
  \defaultfontfeatures{Scale=MatchLowercase}
  \defaultfontfeatures[\rmfamily]{Ligatures=TeX,Scale=1}
\fi
% Use upquote if available, for straight quotes in verbatim environments
\IfFileExists{upquote.sty}{\usepackage{upquote}}{}
\IfFileExists{microtype.sty}{% use microtype if available
  \usepackage[]{microtype}
  \UseMicrotypeSet[protrusion]{basicmath} % disable protrusion for tt fonts
}{}
\makeatletter
\@ifundefined{KOMAClassName}{% if non-KOMA class
  \IfFileExists{parskip.sty}{%
    \usepackage{parskip}
  }{% else
    \setlength{\parindent}{0pt}
    \setlength{\parskip}{6pt plus 2pt minus 1pt}}
}{% if KOMA class
  \KOMAoptions{parskip=half}}
\makeatother
\usepackage{xcolor}
\IfFileExists{xurl.sty}{\usepackage{xurl}}{} % add URL line breaks if available
\IfFileExists{bookmark.sty}{\usepackage{bookmark}}{\usepackage{hyperref}}
\hypersetup{
  pdftitle={Multivariate Data Analysis mit R},
  pdfauthor={Dr.~Uwe Remer},
  hidelinks,
  pdfcreator={LaTeX via pandoc}}
\urlstyle{same} % disable monospaced font for URLs
\usepackage[a4paper]{geometry}
\usepackage{graphicx,grffile}
\makeatletter
\def\maxwidth{\ifdim\Gin@nat@width>\linewidth\linewidth\else\Gin@nat@width\fi}
\def\maxheight{\ifdim\Gin@nat@height>\textheight\textheight\else\Gin@nat@height\fi}
\makeatother
% Scale images if necessary, so that they will not overflow the page
% margins by default, and it is still possible to overwrite the defaults
% using explicit options in \includegraphics[width, height, ...]{}
\setkeys{Gin}{width=\maxwidth,height=\maxheight,keepaspectratio}
% Set default figure placement to htbp
\makeatletter
\def\fps@figure{htbp}
\makeatother
\setlength{\emergencystretch}{3em} % prevent overfull lines
\providecommand{\tightlist}{%
  \setlength{\itemsep}{0pt}\setlength{\parskip}{0pt}}
\setcounter{secnumdepth}{5}

\title{Multivariate Data Analysis mit R}
\usepackage{etoolbox}
\makeatletter
\providecommand{\subtitle}[1]{% add subtitle to \maketitle
  \apptocmd{\@title}{\par {\large #1 \par}}{}{}
}
\makeatother
\subtitle{(und etwas SPSS)}
\author{Dr.~Uwe Remer}
\date{WS 2020/2021}

\begin{document}
\maketitle

{
\setcounter{tocdepth}{2}
\tableofcontents
}
\hypertarget{syllabus}{%
\section{Syllabus}\label{syllabus}}

Campus-Nr.:
\href{https://campus.uni-stuttgart.de/cusonline/pl/ui/$ctx/wbLv.wbShowLVDetail?pStpSpNr=273081}{273081}

Link zum Kurs auf
\href{https://ilias3.uni-stuttgart.de/goto_Uni_Stuttgart_crs_2095425.html}{ILIAS}

\hypertarget{seminarbeschreibung}{%
\subsection{Seminarbeschreibung}\label{seminarbeschreibung}}

Die professionelle Analyse empirischer Daten ist in der
sozialwissenschaftlichen Forschung und in vielen Berufsfeldern von
Sozialwissenschaftlern wichtig. Meist erfordern entsprechende Analysen
auch multivariate Methoden. Deshalb sind Kenntnisse multivariater
Analyseverfahren - deren praktische Anwendung und die Interpretation der
Ergebnisse - wichtige Voraussetzungen, um einerseits empirische Texte
besser verstehen (und kritisieren) sowie andererseits im Studium und
ggf. später im Beruf eigene empirische Analysen durchführen zu können.
Im Seminar sollen die Studierenden mit Hilfe zahlreicher Übungsaufgaben
die dafür notwendigen Kompetenzen entwickeln, indem sie die Verfahren
zunächst vorgestellt bekommen und sich unter Anleitung und später
eigenständig die entsprechenden Lösungswege erarbeiten. Bevor
multivariate Verfahren durchgeführt werden können, sind in der Regel
einige Vorarbeiten an den Daten zu leisten wie z.B. Rekodierungen, die
Bildung von Indizes etc. In der ersten Sitzung werden deshalb kurz die
Verfahren zur Datentransformation wiederholt. Im weiteren Verlauf werden
dann zentrale Analyseverfahren wie Faktorenanalyse, t-Test,
Varianzanalyse und die lineare Regression angewendet und die
Interpretation der Ergebnisse an Beispielen eingeübt.

\hypertarget{r---freie-software-und-state-of-the-art}{%
\subsection{R - Freie Software und ``State of the
Art''}\label{r---freie-software-und-state-of-the-art}}

\begin{quote}
``As is evident in the content of this journal from its inception, and
in books on statistical computing published recently by and for
statisticians, R has come to dominate the development of statistical
software by statisticians. The use of R among political scientists and
others in the social sciences is apparently also on an upwards
trajectory'' --- Altman 2011\footnote{Altman, Micah et al.~(2011): An
  Introduction to the Special Volume on Political Methodology. In:
  Journal of Statistical Software, Vol. 41, Nr.1. DOI:
  \url{http://dx.doi.org/10.18637/jss.v042.i01}}
\end{quote}

Fast zehn Jahre nach dieser Aussage ist R mittlerweile das
Statistikpaket der Wahl für Alle, die quantitative Analysen auf höchstem
Niveau und in größter Flexibilität durchführen wollen und (neben Python)
der Standard in Data Science, Digital Humanities und Computational
Social Science.

R ist eine Programmiersprache zur statistischen Datenanalyse und als
Statistikprogramm R unter der GNU General Public License frei verfügbar.
In der aktuellen Version 4.0.2 läuft es auf Windows, MacOS und Linux.
Das Grundgerüst von R lässt sich durch eine Vielzahl von Paketen
erweitern, mit denen sich unterschiedliche statistische
Problemstellungen bearbeiten lassen.

Ebenso ist es einfach möglich, eigene Erweiterungen zu programmieren. R
ist kommandozeilenbasiert (vergleichbar mit der SPSS-Syntax). Dies hat
den Vorteil, dass die Anwender*In genau wissen muss, was sie überhaupt
vorhat und zu welchem Zweck sie welche statistischen Tests vom Programm
anfordert. Eine besondere Stärke von R liegt in den umfangreichen
Möglichkeiten der Datenvisualisierung.

Zwar ist der Einstieg in R etwas schwieriger, jedoch steigt die
Lernkurve nach einigen Stunden intensiver Beschäftigung (wenn man die
grundsätzliche Logik verstanden hat) stark an. Mit R gibt es dann
keinerlei Beschränkungen mehr, was an Auswertung möglich ist (bis auf
das, was statistisch möglich ist). Zu keinem anderen Statistikprogramm
gibt es eine solche Bandbreite an Unterstützung im Internet
(Video-Tutorials, Mailinglisten, Blogs und Diskussionsseiten). Darüber
hinaus stehen zu R und allen Paketen Handbücher als pdf's frei zur
Verfügung.

Vor allem durch die freie Zugänglichkeit und Unabhängigkeit von
Lizenzen, ist R im universitären Kontext besonders attraktiv. Darüber
hinaus erlernen die Studierenden ein Programm, auf das sie auch nach der
Zeit an der Universität (wo es in der Regel die Lizenzen für die teuren
Statistikprogramme gibt), nutzen können. Die wenigsten Studierende
treffen auf einen Arbeitgeber, bei dem SPSS oder Stata vorhanden ist.
Mit R sind sie dennoch in der Lage statistische Analysen auf höchstem
Niveau durchzuführen, ohne skeptische Verantwortliche („das geht
bestimmt auch mit Excel``) vom Nutzen eines teuren Statistikprogramms
überzeugen zu müssen.

Ein Hinweis zum Schluss: In der R Community gibt es zwei Strömungen:
``\emph{base R}'' (die reine Lehre) und ``\emph{tidyverse}'' (der
Sündenfall, der R massentauglich macht). Sicherlich werden einige von
Ihnen über kurz oder lang gefallen an tidyverse finden. Aus didaktischen
Gründen lernen Sie im Seminar aber base R (trozdem nutzen wir natürlich
eine VIelzahl an Paketen und Befehlen aus dem tidyverse).

\hypertarget{ressourcen}{%
\subsection{Ressourcen}\label{ressourcen}}

R ist Open-Source -- also kostenlose, freie Software. Bitte installieren
Sie auf Ihrem Computer/Laptop schon Mal R. Dazu benötigen Sie zwei Dinge
(bitte auch in dieser Reihenfolge installieren):

\begin{enumerate}
\def\labelenumi{\arabic{enumi}.}
\tightlist
\item
  R: \url{https://cran.r-project.org/bin/windows/base/}
\item
  R-Studio:
  \url{https://rstudio.com/products/rstudio/download/\#download}
\end{enumerate}

\hypertarget{online-ressourcen}{%
\subsubsection{Online Ressourcen}\label{online-ressourcen}}

\begin{itemize}
\tightlist
\item
  R finden Sie im Internet unter \url{https://www.r-project.org}. Über
  den Link CRAN (Comprehensive R Archive Network) können Sie R
  herunterladen.

  \begin{itemize}
  \tightlist
  \item
    Außerdem gibt es ein ausführliches FAQ:
    \url{https://cran.r-project.org/doc/FAQ/R-FAQ.html}
  \item
    Und Sie können herausfinden, welche Pakete es zu welchem Zweck gibt:
    \url{https://cran.r-project.org/web/views/}
  \end{itemize}
\item
  Um das Syntax-, Output- und Datenmanagement mit R zu erleichtern, gibt
  es eine Reihe von grafischen Oberflächen für R. Wir nutzen für das
  Seminar RStudio: \url{https://rstudio.org}
\item
  Tipps und eine Übersicht über die Möglichkeiten mit R findet man bei
  Quick-R unter \url{https://www.statmethods.net}.
\item
  Aktuelle Entwicklungen, Informationen über neue Pakete und besonders
  schöne Beispiele für Analysen gibt es bei den R-Bloggers
  \url{http://www.r-bloggers.com}.
\end{itemize}

\hypertarget{literaturempfehlungen}{%
\subsubsection{Literaturempfehlungen}\label{literaturempfehlungen}}

\hypertarget{r}{%
\paragraph{R}\label{r}}

\begin{itemize}
\tightlist
\item
  Adler, Joseph (2012):
  \href{https://www.oreilly.com/library/view/r-in-a/9781449358204/}{R in
  a Nutshell}. Sebastopol, Calif: O'Reilly Media.
  \href{https://stg.ibs-bw.de/aDISWeb/app?service=direct/0/Home/$DirectLink\&sp=SOPAC02\&sp=SAKSWB-IdNr1619328763}{UB}
\item
  Field, Andy/Miles, Jeremy/Field, Zoë (2012):
  \href{https://uk.sagepub.com/en-gb/eur/discovering-statistics-using-r/book236067}{Discovering
  Statistics Using R}. London: Sage.
  \href{https://stg.ibs-bw.de/aDISWeb/app?service=direct/0/Home/$DirectLink\&sp=SOPAC02\&sp=SAKSWB-IdNr68436977X}{UB}
\item
  Fox, John/Weisberg, Sanford (2018):
  \href{https://uk.sagepub.com/en-gb/eur/an-r-companion-to-applied-regression/book246125}{An
  R companion to applied regression}. Thousand Oaks, Calif: SAGE
  Publications.
  \href{https://stg.ibs-bw.de/aDISWeb/app?service=direct/0/Home/$DirectLink\&sp=SOPAC02\&sp=SAKSWB-IdNr1644736535}{UB}
\item
  Gelman, Andrew/Hill, Jennifer/Vehtari, Aki (2020): Regression and
  Other Stories. Cambridge: Cambridge University Press. doi:
  \href{https://www.cambridge.org/core/books/regression-and-other-stories/DD20DD6C9057118581076E54E40C372C}{10.1017/9781139161879}
\item
  Gelman, Andrew/Hill, Jennifer (2012):
  \href{https://www.cambridge.org/core/books/data-analysis-using-regression-and-multilevelhierarchical-models}{Data
  analysis using regression and multilevel, hierarchical models.}
  Cambridge: Cambridge University Press.
  \href{https://stg.ibs-bw.de/aDISWeb/app?service=direct/0/Home/$DirectLink\&sp=SOPAC02\&sp=SAKSWB-IdNr617446911}{UB}
\item
  Kabacoff, Robert (2015):
  \href{https://www.manning.com/books/r-in-action-second-edition}{R in
  Action. Data Analysis and Graphics with R}. Shelter Island, London:
  Manning.
  \href{https://stg.ibs-bw.de/aDISWeb/app?service=direct/0/Home/$DirectLink\&sp=SOPAC02\&sp=SAKSWB-IdNr798788666}{UB}
\item
  Verzani, John (2014): Using R for introductory statistics. Boca Raton:
  Chapman and Hall.
  \href{https://stg.ibs-bw.de/aDISWeb/app?service=direct/0/Home/$DirectLink\&sp=SOPAC02\&sp=SAKSWB-IdNr786091401}{UB}
\end{itemize}

\hypertarget{wer-etwas-zu-spss-sucht}{%
\paragraph{Wer etwas zu SPSS
sucht\ldots{}}\label{wer-etwas-zu-spss-sucht}}

\begin{itemize}
\item
  Urban, Dieter/Mayerl, Jochen 2018: Angewandte Regressionsanalyse:
  Theorie, Technik und Praxis. Wiesbaden: VS Verlag für
  Sozialwissenschaften.
  \href{https://link.springer.com/content/pdf/10.1007\%2F978-3-658-01915-0.pdf}{pdf
  von SpringerLink via VPN frei verfügbar}
\item
  Backhaus, Klaus/Erichson, Bernd/Plinke, Wulf/Weiber, Rolf 2011:
  \href{https://www.springer.com/de/book/9783662566541}{Multivariate
  Analysemethoden.} Berlin: Springer.
  \href{https://stg.ibs-bw.de/aDISWeb/app?service=direct/0/Home/$DirectLink\&sp=SOPAC02\&sp=SAKSWB-IdNr1026759854}{UB}
\item
  Field, Andy 2018:
  \href{https://www.discoveringstatistics.com/books/dsus/}{Discovering
  Statistics Using SPSS.} London: Sage.
  \href{https://stg.ibs-bw.de/aDISWeb/app?service=direct/0/Home/$DirectLink\&sp=SOPAC02\&sp=SAKSWB-IdNr887803784}{UB}
\end{itemize}

\hypertarget{forschungsdesign-methoden-und-statistik}{%
\paragraph{Forschungsdesign, Methoden und
Statistik}\label{forschungsdesign-methoden-und-statistik}}

\begin{itemize}
\tightlist
\item
  Agresti, Alan/Finlay, Barbara (2009):
  \href{https://www.pearson.com/us/higher-education/program/Agresti-Statistical-Methods-for-the-Social-Sciences-5th-Edition/PGM334444.html}{Statistical
  Methods for the Social Sciences.} Prentice Hall.
  \href{https://stg.ibs-bw.de/aDISWeb/app?service=direct/0/Home/$DirectLink\&sp=SOPAC02\&sp=SAKSWB-IdNr54555618X}{UB}
\item
  Kellstedt, Paul M./Whitten, Guy D. (2013):
  \href{https://www.cambridge.org/core/books/fundamentals-of-political-science-research/D216914982BC901C8E50461818D387A7}{The
  Fundamentals of Political Science Research.} New York: Cambridge
  University Press.
  \href{https://stg.ibs-bw.de/aDISWeb/app?service=direct/0/Home/$DirectLink\&sp=SOPAC02\&sp=SAKSWB-IdNr733649246}{UB}
\item
  Powner, Leanne C. (2015):
  \href{https://uk.sagepub.com/en-gb/eur/book/empirical-research-and-writing}{Empirical
  research and writing. A political science student's practical guide.}
  Los Angeles: Sage CQ Press.
  \href{https://stg.ibs-bw.de/aDISWeb/app?service=direct/0/Home/$DirectLink\&sp=SOPAC02\&sp=SAKSWB-IdNr1616376724}{UB}
\item
  Wolf, Christof/Best, Henning (Hrsg.): Handbuch der
  sozialwissenschaftlichen Datenanalyse. Wiesbaden: VS Verlag für
  Sozialwissenschaften.
  \href{https://link.springer.com/book/10.1007/978-3-531-92038-2}{pdf
  von SpringerLink via VPN frei verfügbar}
\end{itemize}

\hypertarget{formalia}{%
\subsection{Formalia}\label{formalia}}

Das Seminar ``Multivariate Datenanalyse mit R (und etwas SPSS)'' ist
Teil 2 des Moduls 282502 ``Statistik-Software für
Sozialwissenschaftler'' im Studiengang BA Sozialwissenschaften (PO 2012
und PO 2018). Für das Modul erhalten Sie 6 ECTS.

Die Prüfungsleistung für das Modul besteht in einer
lehrveranstaltungsbegleitenden Prüfung (Nr.
\href{https://campus.uni-stuttgart.de/cusonline/pl/ui/$ctx/wbLv.wbShowLVDetail?pStpSpNr=273461}{28252}),
die Sie als kurze Hausarbeit im Umfang von sechs bis acht Seiten in
diesem Seminar erbringen. Abgabefrist für die Hausarbeit ist (Stand
2.11.2020) der 31.03.2021. Die Abgabe erfolgt ausschließlich digital als
pdf und R Code über einen Dateibriefkasten in ILIAS.

Das Seminar hat mit 3 ECTS einen Workload von 90 Stunden, davon 21
Stunden in Präsenz. Präsenz heißt in diesem Semester: ILIAS-Lernmodule,
Lernvideos, Webex-Sitzungen. Darüber hinaus sollten Sie ca. ein bis zwei
Stunden fürs Lesen und eigenständiges Bearbeiten von Übungsaufgaben
einplanen.

Falls Sie einen Termin in der Sprechstunde möchten, vereinbaren wir
einen Termin per Mail. Die Sprechstunde findet dann über Webex statt:
\url{https://unistuttgart.webex.com/meet/uwe.remer}

\hypertarget{sitzungen}{%
\subsection{Sitzungen}\label{sitzungen}}

\hypertarget{sitzung-1---konstituierende-sitzung}{%
\subsubsection{02.11.2020 Sitzung 1 - Konstituierende
Sitzung}\label{sitzung-1---konstituierende-sitzung}}

\hypertarget{sitzung-2---erste-schritte-mit-r}{%
\subsubsection{09.11.2020 Sitzung 2 - Erste Schritte mit
R}\label{sitzung-2---erste-schritte-mit-r}}

\hypertarget{sitzung-3---r-workflow}{%
\subsubsection{16.11.2020 Sitzung 3 - R
Workflow}\label{sitzung-3---r-workflow}}

\begin{itemize}
\tightlist
\item
  Daten einlesen
\item
  Daten und Code handling (R Projekte, Workingdirectory, Ordnerstruktur)
\item
  Code schreiben (Variablenanme, Einrückungen, Kommentare etc.)
\item
  R Markdown
\end{itemize}

\hypertarget{sitzung-4---wiederholung-recodieren-transformieren-indices}{%
\subsubsection{23.11.2020 Sitzung 4 - Wiederholung: recodieren,
transformieren,
Indices}\label{sitzung-4---wiederholung-recodieren-transformieren-indices}}

\hypertarget{sitzung-5---wiederholung-uni--und-bivariate-statistik-aufbereiten-und-ausgeben}{%
\subsubsection{30.11.2020 Sitzung 5 - Wiederholung: uni- und bivariate
Statistik (aufbereiten und
ausgeben)}\label{sitzung-5---wiederholung-uni--und-bivariate-statistik-aufbereiten-und-ausgeben}}

\hypertarget{sitzung-6---mittelwertvergleich-t-test}{%
\subsubsection{07.12.2020 Sitzung 6 - Mittelwertvergleich
(t-Test)}\label{sitzung-6---mittelwertvergleich-t-test}}

\hypertarget{sitzung-7---varianzanalyse-anova}{%
\subsubsection{14.12.2020 Sitzung 7 - Varianzanalyse
(ANOVA)}\label{sitzung-7---varianzanalyse-anova}}

\hypertarget{sitzung-8---explorative-faktorenanalyse-und-reliabilituxe4t}{%
\subsubsection{21.12.2020 Sitzung 8 - Explorative Faktorenanalyse und
Reliabilität}\label{sitzung-8---explorative-faktorenanalyse-und-reliabilituxe4t}}

\hypertarget{sitzung-9---regressionsanalyse-i-einfuxfchrung-bivariater-fall}{%
\subsubsection{11.01.2021 Sitzung 9 - Regressionsanalyse I, Einführung
(bivariater
Fall)}\label{sitzung-9---regressionsanalyse-i-einfuxfchrung-bivariater-fall}}

\hypertarget{sitzung-10---regressionsanalyse-ii-multivariate-regression}{%
\subsubsection{18.01.2021 Sitzung 10 - Regressionsanalyse II,
multivariate
Regression}\label{sitzung-10---regressionsanalyse-ii-multivariate-regression}}

\hypertarget{sitzung-11---regressionsanalyse-iii-kausalituxe4t-und-drittvariablenkontrolle}{%
\subsubsection{25.01.2021 Sitzung 11 - Regressionsanalyse III,
Kausalität und
Drittvariablenkontrolle}\label{sitzung-11---regressionsanalyse-iii-kausalituxe4t-und-drittvariablenkontrolle}}

\hypertarget{sitzung-12---regressionsanalyse-iv-regressionsannahmen-und--diagnostik}{%
\subsubsection{01.02.2021 Sitzung 12 - Regressionsanalyse IV,
Regressionsannahmen und
-diagnostik}\label{sitzung-12---regressionsanalyse-iv-regressionsannahmen-und--diagnostik}}

\hypertarget{sitzung-13---regressionsanalyse-v-moderator--und-mediator-effekte}{%
\subsubsection{08.02.2021 Sitzung 13 - Regressionsanalyse V, Moderator-
und
Mediator-Effekte}\label{sitzung-13---regressionsanalyse-v-moderator--und-mediator-effekte}}

\end{document}
